% Relatório do Projeto Node.js
\documentclass[12pt,a4paper]{article}
\usepackage[brazil]{babel}
\usepackage[utf8]{inputenc}
\usepackage{hyperref}
\usepackage{geometry}
\geometry{a4paper, margin=2.5cm}

\title{Relatório do Projeto Node.js}
\author{Seu Nome}
\date{\today}

\begin{document}

\maketitle

\begin{abstract}
Este relatório apresenta a estrutura, funcionamento e principais características do projeto Node.js desenvolvido para fins de estudo e demonstração de integração com banco de dados, autenticação e manipulação de dados via API REST.
\end{abstract}

\tableofcontents
\newpage

\section{Introdução}
O projeto consiste em uma aplicação Node.js que implementa uma API REST para gerenciamento de autores, livros, resenhas e usuários. O objetivo é demonstrar conceitos de backend, autenticação, persistência de dados e organização de código.

\section{Estrutura do Projeto}
A estrutura de pastas do projeto é a seguinte:

\begin{verbatim}
config/
  database.js
middleware/
  auth.js
models/
  Author.js
  Book.js
  Review.js
  User.js
public/
  app.js
  index.html
routes/
  auth.js
  authors.js
  books.js
  reviews.js
server.js
package.json
README.md
\end{verbatim}

\section{Descrição dos Principais Arquivos e Pastas}
\begin{itemize}
  \item \textbf{config/database.js}: Configuração da conexão com o banco de dados.
  \item \textbf{middleware/auth.js}: Middleware para autenticação de usuários.
  \item \textbf{models/}: Modelos Mongoose para as entidades Author, Book, Review e User.
  \item \textbf{public/}: Arquivos públicos, incluindo frontend simples (HTML/JS).
  \item \textbf{routes/}: Rotas da API para autenticação, autores, livros e resenhas.
  \item \textbf{server.js}: Arquivo principal que inicializa o servidor Express.
  \item \textbf{package.json}: Gerenciamento de dependências e scripts do projeto.
\end{itemize}

\section{Tecnologias Utilizadas}
\begin{itemize}
  \item Node.js
  \item Express.js
  \item MongoDB (via Mongoose)
  \item JWT para autenticação
  \item HTML/CSS/JS para frontend básico
\end{itemize}

\section{Como Executar}
\begin{enumerate}
  \item Instale as dependências: \\ \texttt{npm install}
  \item Configure o banco de dados em \texttt{config/database.js} e variáveis de ambiente.
  \item Inicie o servidor: \\ \texttt{npm start} ou \texttt{node server.js}
  \item Acesse a aplicação via navegador ou ferramentas como Postman.
\end{enumerate}

\section{Considerações Finais}
O projeto serve como base para aplicações Node.js com autenticação, persistência de dados e organização modular. Pode ser expandido para incluir mais funcionalidades, testes automatizados e interface de usuário aprimorada.

\end{document} 